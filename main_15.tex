\documentclass{article}
\usepackage[utf8]{inputenc}

\title{Assignment 15}
\author{Xiaoting Li (xil139) \\
Ziyu Zhang (ziz41) \\
Deniz Unal (des2014)}
\date{}

\begin{document}

\maketitle

\noindent
\textbf{25. Consider the following problem. The input is n disjoint line segments contained in an $L$ by $L$ square $S$ in the Euclidean plane. The goal is to partition S into convex polygons so that every polygon $P$ , the interior of polygon $P$ intersects at most one line segment. So it is ok for a line segment to be in multiple polygons, but each polygon interior can intersect at most one line segment. Its alway ok for many different line segments to intersect the boundary of a polygon. \\ \newline 
Consider the following algorithm that starts with the polygon $S$. Let $\pi$ be a random permutation of the line segments. While there is a polygon $P$ that contains more than one line segment, let $\l$ be the first line segment in the $\pi$ order that intersects $P$. Then cut $P$ into two polygons using the linear extension of $\l$ (so you extend the line segment $\l$ into a line and then use that to cut $P$ into two polygons). Show that the expected number of resulting polygons is $O(n log n)$.} \\ \newline
Answer: From the hint, we know that before $u$ hits $v$, it needs to hit $N(u, v)$ segments. We have $C_{u,v} = 1$ if $u$ hits $v$, and we have $C_{u,v} = 0$ if $u$ doesn't hit $v$. So it means that if we have $C_{u,v} = 1$ then in a random permutation $\pi$, we need to have $u$ comes before all of these $N(u, v)$ segments and the $v$ segment. The probability of such order in the permutation is $\frac{1}{N(u, v) + 1}$. Since $u$ hits $v$ only when we have such order in the permutation, so we have $E[C_{u,v}] \leq \frac{1}{N(u, v) + 1}$. Let $n$ be the number of non-intersecting segments in $S$. The expected number of resulting polygons depends on the number of non-intersecting line segments as well as the expected number of fragments generated by the intersection. So we have the expected number of resulting polygons as 
$$n + E[\sum_{u,v}C_{u,v}]$$
Then we have
$$n + E[\sum_{u, v}C_{u,v}] \leq n + E[\sum_{u, v}\frac{1}{N(u, v) + 1}]$$
Then we have
$$n + E[\sum_{u, v}C_{u,v}] \leq n + E[\sum_{N(u, v) = 1}^{n - 1}\frac{1}{N(u, v) + 1}]$$
Since every time we extend a line segment, we extend it into two directions, so there are at most two segments of the same $N(u, v)$. So we have
$$n + E[\sum_{u, v}C_{u,v}] \leq n + E[\sum_{u, v}\frac{1}{N(u, v) + 1}] \leq n + E[\sum_{u, v}\frac{2}{N(u, v) + 1}]$$
Since 
$$ n + E[\sum_{u, v}\frac{2}{N(u, v) + 1}] = n + 2nlg{n}$$
So we have the expected number of resulting polygons as $O(nlg{n})$.

\end{document}
