\documentclass{article}
\usepackage[utf8]{inputenc}
\usepackage{amsmath}

\title{Assignment 10}
\author{Xiaoting Li (xil139) \\
Ziyu Zhang (ziz41) \\
Deniz Unal (des2014)}
\date{February 2019}

\begin{document}

\maketitle

\noindent
\textbf{18.
(a) Consider the problem of finding a minimum weight arborescence of an directed edge-weight graph with designed root $r$. Give a polynomial sized integer linear programming formulation.} \\ \newline
Answer: For minimum weight arborescence problem, we need to make sure that there exists a path from root $r$ to all other vertices. And we need to make sure the arborescence is a spanning tree that contains no cycle. In order to get a polynomial sized integer linear programming formulation, we need to convert this problem to a network flow problem to reduce the number of constraints. We add $f_{ij}^{k}$ for every $(i, j)\in E$ and $k\in V$. We define $c_{ij}$ as the weight of edge $(i, j)$. So the objective of the problem is to minimize $\sum_{(i,j)\in E}f_{ij}c_{ij}$. The constraints are: \\\newline
(1) $f_{ij}^{k}\in \{{0, 1}\}$, for all $k\in V - \{{r}\}$\\ \newline
(2) $\sum_{j:(r,j)\in E} f_{ij}^{k} = 1$, for all $k\in V - \{{r}\}$ \\ \newline
(3) $\sum_{i: (i,v)\in E}^{k} = \sum_{j:(v, j)\in E}^{k}$, for all $v, k\in V - \{{r}\}$ \\ \newline
If an edge is in arborescence, $f_{ij}^{k}$ is 1. We have $VE$ variables, and we have $VE + V + V^2$ constraints. So we have a polynomial sized integer linear programming formulation for minimum weight arborescence problem.\\ \newline
\textbf{(b) Consider the problem of finding a minimum weight spanning tree of an undirected edge-weight graph. Give a polynomial sized integer linear programming formulation.}\\ \newline
Answer: For minimum weight spanning tree, we need make sure there is $|V| - 1$ edges in the spanning tree and there is no cycle in the spanning tree.  Like (a), we convert this problem into a network flow problem to get polynomial sized linear programming. We add $f_{ij}^{k}$ for every $(i, j)\in E$ and $k\in V$. We define $c_{ij}$ as the weight of edge $(i, j)$. So the objective of the problem is to minimize $\sum_{(i,j)\in E}f_{ij}c_{ij}$. The constraints are: \\ \newline
(1) $f_{ij}^{k}\in \{{0, 1}\}$, for all $k\in V$\\ \newline
(2) $\sum_{(i,j)\in E}f_{ij}^{k} \leq k - 1$, for all $k\in V$\\ \newline
(3) $\sum_{i: (i,v)\in E}^{k} = \sum_{j:(v, j)\in E}^{k}$, for all $v, k\in V$ \\ \newline
If an edge is in the spanning tree, $f_{ij}^{k}$ is 1. We have $VE$ variables, and we have $VE + V + V^2$ constraints. So we have a polynomial sized integer linear programming formulation for minimum weight spanning tree.\\ \newline
\textbf{(c) Consider the problem of finding a minimum weight spanning tree of an undirected edge-weight graph $G = (V, E)$. Give an integer linear programming formulation using the following intuition, and prove that your formulation is correct: There is an indicator $0/1$ random variable for each edge. You must choose at least $n - 1$ edges ($n$ is number of vertices in the graph). For each subset $S$ of $k$ vertices, you can choose at most $k - 1$ edges connecting vertices in $S$. Explain why the size of this linear program can be exponential in the size of the graph.} \\\newline
Answer: The constraints of minimum weight spanning tree are: \\
(1) There are $|V| - 1$ edges in the spanning tree: $$\sum_{(i,j)\in E} (i,j) = |V| - 1$$
(2) There are no cycles in the spanning tree: $$\sum_{(i,j)\in E, i\in S, j\in S}(i, j) \leq |S| - 1, \forall S\subseteq{V}$$ 
(3) If an edge $(i,j)$ is in the spanning tree, it equals to 1, else it equals to 0: $$(i,j)\in \{0,1\}$$
The size of the second constraint is exponential. The second constraint makes sure there is no cycle in the spanning tree. It means when we pick $n$ vertices randomly in the spanning tree $n\leq |V|$, there is no cycle in these $n$ vertices. So the number of possibilities is the sum of combinations, $\sum_{1\leq n \leq |V|}\binom{n}{|V|}$, which equals to $2^n$. $n$ is related with the size of the graph. So we can say that size of this linear program can be exponential in the size of the graph. \\ \newline
\textbf{(d) The Ellipsoid algorithm can solve a rational linear program in time polynomial in the number of variables, even if the number of constraints is exponential, as long as there is a polynomial time algorithm for the separation problem. In the separation problem, the input is a value/number $\hat{x_e}$ for each the variable $x_e$ in the linear program. The output should be a constraint in the linear program that is violated by these values, if such a constraint exists. Give a polynomial time algorithm for the separation problem for the linear program with exponentially many variables you designed in the pervious subproblem.} \\
Answer: In order to find a cut that violates a constraint, we need to check if the min $s - t$ cut is smaller than $|V|$. Assume there is a set $S \subset V$ with $x(E(S)) > |S| - 1$ where $x(E(S))$ denotes $\sum_{e\in E(S)} \hat{x_e}$. We consider the following network N. The vertices of N are V plus a source vertex s and a sink vertex t. The edges in N are E plus an edge between each vertex in V and s, and an edge between each vertex in V and t. Each edge $e \in E$ is given a capacity equal to $\hat{x}_e/2$. Each edge between a vertex $v \in V$ and $t$ is given a capacity of 1. Each edge between a vertex $v \in V$ and $s$ is given a capacity of $x(f(v))/2$ where $f(v)$ denotes the set of edges that have exactly one endpoint in v. Let $S \cup {s}$ be an $s-t$ cut in N and its value is: \\
\begin{flalign*}
|S| + \sum_{v \in V \setminus S} x(f(v)) / 2 \, + \, \sum_{e\in f(S)} \hat{x}_e / 2 \, = \, |S| + x(E(V)) - x(E(S))
\end{flalign*}
Thus we found a cut that has value smaller than $|V|$.
\end{document}
