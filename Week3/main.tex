\documentclass{article}
\usepackage[utf8]{inputenc}

\title{Assignment 3}
\author{lxtlouise2014 }
\date{January 2019}

\begin{document}
\maketitle
\textbf{4. Problem 26-5 from the CLRS text.} \\
\noindent
Answer: \\
(a) The capacity of a cut from G is less or equal to the sum of capacity of all edges, and for each edge the max capacity is less or equal to C, so the cut is less or equal to all the edges with C capacity, which is C \mid E \mid.\\


\maketitle
\textbf{5. Problem 26-6 from the CLRS text.} \\
\noindent
Answer: \\
(a) If M is a matching and P is an augmenting path with respect to M, it means that there are $k$ edges in M and there $k + 1$ edges not in M. Since we are calculating the symmetric difference between M and P, it is easy for us to learn that there are $2k + 1$ edges in $M\cup P$ and there are $k$ edges in $M\cap P$. Therefore, we can get $$|M \oplus P| = |M\cup P| - |M\cap P| = |M| + 2k + 1 - 2k = |M| + 1$$ And we can get $$M \oplus (P_1\cup P_2 \cup ... \cup P_k )  = |M\cup P_1\cup P2\cup ... \cup P_k| - |M\cap P1\cap P_2\cap ... \cap P_k| = $$ $$|M|\cup |P_1|\cup |P_2|\cup ... \cup |P_k| -  |M\cap P1\cap P_2\cap ... \cap P_k| = $$ $$|M| + (2k_1 + 1) + (2k_2 + 1) + ... + (2k_k + 1) - (2k_1 + 2k_2 + ... + 2k_k) = $$ $$|M| + k$$ \\
(b) Since $G^{'} = (V, M\oplus M^{*})$, at least 2 of the edges come from the same matching if each vertex has more than 3 degrees. However, this contradicts with the definition of matching. No two edges in a matching have a common vertex. Therefore, every vertex in $G^{'}$ has at most 2 degrees. Since every vertex in $G^{'}$ has at most 2 degrees, it means there is no path in the graph that has repeated vertices. Therefore, $G^{'}$ is a disjoint union of simple paths and cycles. Since two edges with the same vertex cannot appear in the same matching and $G^{'} = (V, M\oplus M^{*})$, so edges in each such simple path or cycle belong alternately to $M$ or $M^{*}$. Since $|M| < |M^{*}|$ and we've already known that edges in $G^{'}$ are alternated between $M$ and $M^{*}$, it means that those vertices that are not in $M$ but in $M^{*}$ are unmatched points for $M$ and the number of vertex-disjoint augmenting paths with respect to $M$ is $|M| - |M^{*}|$.\\ \newline
(c) Since $P$ is an augmenting path with respect to $M^{'}$ and $M^{'} = M\oplus (P_1\cup P_2\cup ... \cup P_k)$, $P$ contains edges from $M$ or from $(P_1\cup P_2\cup ... \cup P_k)$ and the length of P must be at least $l$. The end vertices of P are unmatched in $M^{'}$, which means they're unmatched in $M$ as well. And the edges of $P$ that contains one of the end vertices cannot be in $(P_1\cup P_2\cup ... \cup P_k)$ or $M$ since $P$ is vertex-disjoint with $(P_1\cup P_2\cup ... \cup P_k)$, or it contradicts with the fact that $(P_1\cup P_2\cup ... \cup P_k)$ is the maximum set of of vertex-disjoint augmenting paths of length $l$ with respect to $M$.\\ \newline
(d) Any edge in $M\oplus M^{'}$ is either in $M$ or $M^{'}$. Since $M^{'} = M\oplus (P_1\cup P_2\cup ... \cup P_k)$, it means that any edge in $M^{'}$ is either in $M$ or $P_1\cup P_2\cup ... \cup P_k$. So the symmetric difference between $M$ and $M^{'}$ is $P_1\cup P_2\cup ... \cup P_k$. And $A = M\oplus M^{'}\oplus P$. Therefore, $A = (P_1\cup P_2\cup ... \cup P_k)\oplus P$. From (a) we learn that the size of $(P_1\cup P_2\cup ... \cup P_k)$ is k. Also, the $(P_1\cup P_2\cup ... \cup P_k)$ is the maximum set of of vertex-disjoint augmenting paths of length $l$. If $P$ has the length of $l$, it contradicts with the fact that $A = (P_1\cup P_2\cup ... \cup P_k)\oplus P$ (there are $k+1$ augmenting paths). Therefore, we have $|A| \geq (k + 1)l$ and the length of $P$ has more than $l$ edges. \\\newline
(e) Suppose we have a maximum matching $M^{*}$ that has more than $|M| + |V|/(l + 1)$ edges. Based on what we got from (b), we learn that the number of vertex-disjoint augmenting paths with respect to $M$ is $|M| - |M^{*}|$, which is $|V|/(l + 1)$. Since the length of the shortest augmenting path with respect to M is $l$, it means that each path has $l + 1$ vertices. Also, the augmenting paths are vertex-disjoint, which means there are more than $|V|$ distinct vertices incident with all of the augmenting paths and it is a contradiction. Therefore, the size of the maximum matching is $|M| + |V|/(l + 1)$.\\ \newline
(f) Based on what we got from (b), we learn that the number of vertex-disjoint augmenting paths with respect to $M$ is $|M| - |M^{*}|$. And from (b), we also learn that graph $G^{'}$ has $|V|$ vertices. So we learn that $|M| - |M^{*}| \leq \sqrt{V}$, which means $M$ can grow no more than $\sqrt{V}$ after iteration number $\sqrt{V}$.\\ \newline
(g) If we modify BFS to traverse the graph, it takes at most $O(E)$ to alternate between $M$ and $E - M$, which means it takes $O(E)$ to find an augmenting path. From (f) we learn that the number of repeat loop iterations in the algorithm is at most $2\sqrt{V}$, which means each time it takes $O(\sqrt{V})$ to find a vertex disjoint shortest path. And we learn that that runs in $O(E)$ time to find a maximal set of vertex disjoint shortest paths, so we can say that the total running time of HOPCROFT-KARP is $O(\sqrt{V}E)$. \\


\end{document}
