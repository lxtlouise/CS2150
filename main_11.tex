\documentclass{article}
\usepackage[utf8]{inputenc}
\usepackage{amsmath}

\title{Assignment 11}
\author{Xiaoting Li (xil139) \\
Ziyu Zhang (ziz41) \\
Deniz Unal (des2014)}
\date{February 13 2019}

\begin{document}

\maketitle

\noindent
\textbf{19. Consider the problem of constructing maximum cardinality bipartite matching ( See section 26.3 in the CLR text). Consider the natural integer linear program that you derived in a prior homework problem.} \\
\textbf{(a) Construct the dual linear program.} \\ \newline
Answer: We construct the natrual integer linear program for maximum cardinality bipartite matching as 
\begin{flalign*}
\text{max } &\sum_{e \in E} x_e \qquad \text{s.t.} \\
&\forall v \in V, \sum_{e \, \text{adjacent to} \, v} x_e \leq 1 \\
&\forall e \in E, \; x_e \in \{0, 1\}
\end{flalign*}
To get the dual of the problem, we need to construct constraints from the objective in the primal and construct objective from the constraints in the primal. Let's have a variable $\alpha_{v}$ for the second constraint in the primal linear program. We can get the dual of this problem as below: 
\begin{flalign*}
\text{min } &\sum_{v \in V} \alpha_v \qquad \text{s.t.} \\
&\forall (u, v) \in E, \quad   \alpha_u + \alpha_v \geq 1 \\
&\forall v \in V, \; \alpha_v \in \{0, 1\}
\end{flalign*} \\ \newline
\textbf{(b) Give a natural English interpretation of the dual problem (e.g. similar to how we interpreted the dual of diet problem as the pill problem)} \\ \newline
The dual problem is to minimize the number of vertices which are the endpoints of more than one edge in the bipartite graph. \\ \newline
\textbf{(c) We previously showed that there was always an integer optimal solution (so all values are 0 or 1) to the primal linear program. Does the dual problem always have an integer optimal solution? Justify your answer.} \\ \newline
In bipartite graphs, every basic feasible solution of dual problem of is integral since every optimal solution must have $\alpha_v \geq 1 \,\, \forall v \in V$ and so every optimal basic feasible solution has  $\alpha_v \in \{0, 1\} \,\, \forall v \in V$, thus it is a feasible solution for integer-dual problem.\\ \newline
\textbf{(d) Explain how to give a simple proof that a graph doesn’t have a matching of a particular size based on the dual linear program. You should be able to come up with a method that would convince someone who knows nothing about linear programming.
} \\ \newline
Finding a matching of particular size is equivalent to find particular number of points that have more than one edge. However, the graph may not have that number of points which have more than on edge. \\ \newline

\end{document}
