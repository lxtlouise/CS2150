\documentclass{article}
\usepackage[utf8]{inputenc}
\usepackage{url}

\title{Assignment 21}
\author{Xiaoting Li (xil139) \\
Ziyu Zhang (ziz41) \\
Deniz Unal (des2014)}
\date{March 8 2019}

\begin{document}

\maketitle

\noindent
\textbf{32. Problem 8-6 from the CLRS text.}\\ \newline
\textbf{33. We consider the distributed consensus problem as discussed in class and in the notes \url{http://homepage.divms.uiowa.edu/~ghosh/16612.week11.pdf}.}\\ \newline
\textbf{(a) Give a message passing algorithm that achieves distributed consensus if there are no processors failures.} \\ \newline
Answer: Since we need to achieve distributed consensus if there are no processors failures, we need to make sure that the algorithm meets the specification on termination, validity, and agreement. Let's say we have an iterative algorithm works as below. Each processor picks a favorite value, put this value in the message, and broadcast the message to all other processors. Since all messages will eventually be delivered, so when each processor receives the broadcast messages from other processors, it can change the favorite value based on the majority of the favorite values. Then repeat the same steps to do the broadcast again. We need to make sure the algorithm can get to termination. So we set a threshold $t$. As long as a processor achieves up to $t$ messages that have the same favorite value from at least $n/2$ processors ($n$ is the total number of processors), it means that the system achieves agreement. Then the process broadcast this value again and commits to the outcome. Using such algorithm, we make sure that every processors in the system will eventually decide and terminate, at the same time achieves agreement with valid values  \\ \newline
\textbf{(b) Now assume that any 1 processor can experience a failure at any time (but $n-1$ of the processors will always work properly). You can think of a failure meaning that the machine is just turned off; The failed processors doesn’t do anything malicious. Show that there is no deterministic algorithm/protocol that can achieve distributed consensus in this setting.} \\ \newline 
Answer: \\ \newline
\textbf{(c) Now assume that the transport layer connection additionally guarantees the in order delivery of messages between processors, like TCP. So if processor $i$ sends processor $j$ three messages, they will arrive at $j$ in the order that $i$ sent them. There is still no upper bound on the arrival time. Also there is no guarantee about the order of arrival of messages either sent to different processors, or sent from different processors. Now prove or disprove that there is a deterministic protocol that will achieve distributed consensus with up to one processor failure.}\\ \newline
Answer:

\end{document}
