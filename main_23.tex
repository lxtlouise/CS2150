\documentclass{article}
\usepackage[utf8]{inputenc}
\usepackage{mathtools}
\DeclarePairedDelimiter\floor{\lfloor}{\rfloor}

\title{Assignment 23}
\author{Xiaoting Li (xil139) \\
Ziyu Zhang (ziz41) \\
Deniz Unal (des204)}
\date{March 20 2019}

\begin{document}

\maketitle

\noindent
\textbf{36. Let $P$ be a problem. The worst case time complexity of $P$ is $O(n^2)$. The worst case time complexity of $P$ is $\omega(nlogn)$. Let $A$ be an algorithm that solves $P$ . For each of the following statements, state whether the statement is logically implied by the above information, and state whether the statement is logically consistent with the above information. Justify your answers.} \\ \newline
\textbf{(a)} This one is not logically consistent with the above information. Since $A$ is an algorithm that solves $P$. It means we can call $A$ in $P$. And since the worst case time complexity of $P$ is $O(n^2)$, the worst case time of $A$ cannot be higher than $O(n^2)$.\\ \newline
\textbf{(b)} This one is not logically consistent with the above information Since $A$ is an algorithm that solves $P$. It means we can call $A$ in $P$. And since the worst case time complexity of $P$ is $\omega(nlogn)$, the worst case time of $A$ cannot be higher than $\Omega(nlogn)$. \\ \newline
\textbf{(c)} \\ \newline
\textbf{(d)} This one is logically consistent with the above information. Since $A$ is an algorithm that solves $P$. It means we can call $A$ in $P$. And since the worst case time complexity of $P$ is $O(n^2)$, and the worst case time complexity of $A$ is $O(nlogn)$.\\ \newline
\textbf{(e)} This one is logically consistent with the above information. $\Omega(nlogn)$ is tigher bound than $\omega(nlogn)$\\ \newline
\textbf{(f)} \\ \newline
\textbf{(g)} This one is logically consistent with the above information.\\ \newline
\textbf{(h)} This one is logically consistent with the above information.\\ \newline
\textbf{(i)} This one is logically consistent with the above information.\\ \newline
\textbf{37. Consider the element uniqueness problem. In this problem the input is $n$ real numbers $x_1,..., x_n$, and the problem is to determine whether there exists an $i$ and a $j$ such that $i\neq j$ and $x_i = x_j$.}\\\newline
\textbf{(a) Give an input distribution such every comparison-based deterministic correct algorithm requires $\Omega(nlogn)$ comparisons in expectation.} \\ \newline
Answer: If the input distribution is uniform distribution, then every comparison-based deterministic correct algorithm requires $\Omega(nlogn)$ comparisons in expectation. If the input distribution is uniform distribution, we have the average case of number of comparison as at least $\floor{log_2n!}$. We can use recursive call tree to prove that the average depth of the tree is at least $\floor{log_2n!}$. If the tree is balanced, then it's straightforward that the average height is $\floor{log_2n!}$. If it is unbalanced, we can modify the tree and also get the average depth as at least $\floor{log_2n!}$. So we have $\Omega(nlogn)$ comparisons in expectation.

\end{document}
