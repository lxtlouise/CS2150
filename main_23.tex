\documentclass{article}
\usepackage[utf8]{inputenc}

\title{Assignment 23}
\author{Xiaoting Li (xil139) \\
Ziyu Zhang (ziz41) \\
Deniz Unal (des204)}
\date{March 20 2019}

\begin{document}

\maketitle

\noindent
\textbf{36. Let $P$ be a problem. The worst case time complexity of $P$ is $O(n^2)$. The worst case time complexity of $P$ is $\omega(nlogn)$. Let $A$ be an algorithm that solves $P$ . For each of the following statements, state whether the statement is logically implied by the above information, and state whether the statement is logically consistent with the above information. Justify your answers.} \\ \newline
\textbf{(a)} This one is not logically consistent with the above information. Since $A$ is an algorithm that solves $P$. It means we can call $A$ in $P$. And since the worst case time complexity of $P$ is $O(n^2)$, the worst case time of $A$ cannot be higher than $O(n^2)$.\\ \newline
\textbf{(b)} This one is not logically consistent with the above information Since $A$ is an algorithm that solves $P$. It means we can call $A$ in $P$. And since the worst case time complexity of $P$ is $\omega(nlogn)$, the worst case time of $A$ cannot be higher than $\Omega(nlogn)$. \\ \newline
\textbf{(c)} \\ \newline
\textbf{(d)} This one is logically consistent with the above information. Since $A$ is an algorithm that solves $P$. It means we can call $A$ in $P$. And since the worst case time complexity of $P$ is $O(n^2)$, and the worst case time complexity of $A$ is $O(nlogn)$.\\ \newline
\textbf{(e)} This one is logically consistent with the above information. $\Omega(nlogn)$ is tigher bound than $\omega(nlogn)$\\ \newline
\textbf{(f)} \\ \newline
\textbf{(g)} This one is logically consistent with the above information.\\ \newline
\textbf{(h)} This one is logically consistent with the above information.\\ \newline
\textbf{(i)} This one is logically consistent with the above information.\\ \newline
\textbf{37. Consider the element uniqueness problem. In this problem the input is $n$ real numbers $x_1,..., x_n$, and the problem is to determine whether there exists an $i$ and a $j$ such that $i\neq j$ and $x_i = x_j$.}

\end{document}
