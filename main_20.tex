\documentclass{article}
\usepackage[utf8]{inputenc}
\usepackage{amsmath}
\usepackage{mathtools}

\title{Assignment 20}
\author{Xiaoting Li (xil139) \\
Ziyu Zhang (ziz41) \\
Deniz Unal (des2014)}
\date{March 6 2019}

\begin{document}

\maketitle

\noindent
\textbf{31. (a)Consider the problem where the input is a sorted array $A$ containing $n$ real numbers and a real number $x$, and the output is an integer $k$ such that if $x$ is in $A$ then it must lie between positions $k$ and $k + \sqrt{n}$. Assume that each element of $A$ and $x$ are independently and uniformly distributed in the interval [0, 1]. Show that the following algorithm solves this problem in $O(1)$ average case time.} \\ \newline 
Answer: Let $X$ be the number of comparisons to find $k$, $X_i$ means the number of comparisons to find $k$ for number $i$. Since it is Bernoulli trials, the expectation of $E(X_i) = 1/(1/\sqrt{n}) = \sqrt{n}$. Using Bernoff Bound, We have
\begin{align*}
&E(X) = Prob(X \leq \sqrt{n})\cdot \sqrt{n} + Prob(X > \sqrt{n})\cdot(2\sqrt{n} - 1) \\
&\leq Prob(X \leq \sqrt{n})\cdot \sqrt{n} + Prob(X > \sqrt{n}) \cdot (2\sqrt{n} - 1)\\
&\leq Prob(X \leq \sqrt{n})\cdot \sqrt{n} + Prob(X > \sqrt{n}) \cdot 2\sqrt{n}\\
&\leq Prob(X \leq \sqrt{n})\cdot \sqrt{n} + Prob\{X - \sqrt{n}\geq r\}\cdot 2\sqrt{n}\\
&\leq 2e^{-r^2/2n}
\end{align*}
Let's set $r$ as a small number greater than 0, $r = \sqrt{ln\frac{1}{\sqrt{n}}}\cdot \sqrt{n}$. Plug $r$ into the equation above, and we can get 
\begin{align*}
&E(X) = Prob(X \leq \sqrt{n})\cdot \sqrt{n} + Prob(X > \sqrt{n})\cdot(2\sqrt{n} - 1)\\
&\leq 2e^{-r^2/2n}\\
&=2\cdot2\cdot e^{-1/2}\\
&= O(1)
\end{align*}\newline
\textbf{(b) Explain how to use the above algorithm to obtain an algorithm with $O(log log n)$ average case running time for the searching problem (finding the location of the element in $A$ whose value is closest to $x$). Again assume that each element of $A$ and $x$ are independently and uniformly distributed in the interval [0, 1].}\\ \newline
Answer: Build a binary search tree $T$ using the sorted array A. And then search $k$ in the following way: start with last = $x*n$, if in the search tree, $T[last]$ is smaller than $x$, than jump to left tree at level $T[last] + log(n)$ to do the comparison until we find value larger than $x$, then return that position as $k$. 


\end{document}
