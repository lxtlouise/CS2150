\documentclass{article}
\usepackage[utf8]{inputenc}

\title{Assignment 26}
\author{Xiaoting Li (xil139) \\
Ziyu Zhang (ziz41) \\
Deniz Unal (des204)}
\date{March 27 2019}

\begin{document}
\maketitle
\noindent
\textbf{43. A dominating set in a graph is a subset $D$ of the vertices such that every vertex in the graph either belongs to $D$ or is adjacent to $D$. Consider the optimization problem of finding a minimum cardinality dominating set in graphs with maximum degree of three. Give a fixed parameter tractable (FPT) algorithm where the parameter k is the cardinality of the minimum cardinality dominating set. Your algorithm should be based on exhaustive search with backtracking.} \\ \newline
Answer: We can convert this problem to looking for a min set cover problem. Given the input graph $G = (V, E), V = {1, 2, ... n}$, we can construct an set cover instance $(U, S)$, $U$ is $V$ and $S = {S_1, S_2, ... S_n}$ such that $S_v$ consists of vertex $v$ and all the adjacent vertices to $v$ in graph $G$. The size of minimum cardinality dominating set is the size of minim set cover for $(U, S)$. Next, we use exhaustive search with backtracking to find min set cover with maximum degree of three in the set cover instance. Each time the algorithm picks a vertex $v$ and compute set $S_v$, then it picks an undiscovered vertex $u$ that is adjacent to $v$ and compute set $S_u$. The algorithm repeats this step until there is no more undiscovered vertex of this search branch. Then it checks whether this is the solution. If not, it backtracks to pick another undiscovered adjacent vertex to $v$ and repeat this process. If we find a min vertex cover with maximum degree of three in the set cover instance, say the solution is $S = {S_2, S_4}$ (min set cover with maximum degree of three), then minimum cardinality dominating set in graphs with maximum degree of three is $D= {2, 4}$. The time of this algorithm is $k \cdot 2^k \cdot n$, which means the algorithm is FPT.\\ \newline
\textbf{44. A complete bipartite cover of a graph $G$ is a collection of complete bipartite subgraphs $H_1, ... H_k$ of $G$ that every edge in $G$ is in at least one $H_i$. Consider the optimization problem of finding a minimum cardinality complete bipartite cover of $G$ (so we want to minimize $k$). First give a kernelization algorithm for this problem, where the parameter $k$ is the minimum cardinality of a complete bipartite graph cover. Then give a fixed parameter tractable algorithm for the problem.} \\ \newline
Answer: 

\end{document}
