\documentclass{article}
\usepackage[utf8]{inputenc}
\usepackage{amsmath}

\title{Assignment 30}
\author{Xiaoting Li (xil139) \\
Ziyu Zhang (ziz41) \\
Deniz Unal (des204)}
\date{April 5 2019}

\begin{document}

\maketitle

\noindent
\textbf{54. Problem 15-3 from the text. The purpose of this problem is to give you some practice coming up with a dynamic program.} \\ \newline
Answer: We first order $n$ points in increasing order based on their x-coordinate. From left to right, we denote the points as $p_1, p_2,..., p_n$. We also denote the euclidean distance between $p_i$ and $p_j$ as $d(i, j)$, and denote the length of the shortest bitonic path between $p_i$ and $p_j$ as $b(i, j)$. From the hint, we can think this as a binary tree, so level to level, we have $b(1, 2), b(1, 3) b(2, 3), b(1, 4), b(2, 4), ...,b(i, j)$. We know that bitonic tours are tours that start at the leftmost point and go strictly rightward to the rightmost point, and then go strictly leftward back to the starting point. So all points that are left to the $p_j$ are visited by the tour. So when $i < j - 1$, in order to get $b(i,j)$, we need to know $b(i, j-1)$ and $d(j-1, j)$. We have $b(i,j) = b(i, j-1) + d(j-1, j)$. The baseline is $b(1,2) = d(1,2)$, baseline is simply the euclidean distance between $p_1$ and $p_2$. When $i = j - 1$, we know that there must be some path that gets to $p_k$ and from $p_k$ that gets to $p_j$, and $1 < k < j - 1$. So at this time we need to get the shortest path among all of such points. We can write the dynamic algorithm as 
\begin{flalign*}
&b(1,2) = d(1,2) \\
&b(i,j) = b(i, j-1) + d(j-1, j), \quad i < j - 1 \\
&b(i,j) = min\{b(k, j) + d(k, j)\}, \quad i = j - 1, 1 < k < j - 1
\end{flalign*}
No matter $i < j - 1$ or $i = j - 1$, it takes two for-loops to do the calculation. So the algorithm is $O(n^2)$.
\\ \newline
\textbf{55. The goal in this problem is to give polynomial time approximation scheme (PTAS) for the parallel machine scheduling problem described in problem 35-5 in CLR, for the case that the number of processors/machines $m = 3$, using the following approach:} \\ \newline
\textbf{• First give a dynamic programming algorithm whose time in polynomial in $n$, and $P$, where $P$ is the total aggregate processing times of all of the jobs. So this is a pseudo-polynomial-time algorithm, but not a polynomial time algorithm.} \\ \newline
\textbf{• Then explain how convert this pseudo-poly-time dynamic program into PTAS.}

\end{document}
