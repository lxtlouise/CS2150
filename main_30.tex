\documentclass{article}
\usepackage[utf8]{inputenc}
\usepackage{amsmath}

\title{Assignment 30}
\author{Xiaoting Li (xil139) \\
Ziyu Zhang (ziz41) \\
Deniz Unal (des204)}
\date{April 5 2019}

\begin{document}

\maketitle

\noindent
\textbf{54. Problem 15-3 from the text. The purpose of this problem is to give you some practice coming up with a dynamic program.} \\ \newline
Answer: We first order $n$ points in increasing order based on their x-coordinate. From left to right, we denote the points as $p_1, p_2,..., p_n$. We also denote the euclidean distance between $p_i$ and $p_j$ as $d(i, j)$, and denote the length of the shortest bitonic path between $p_i$ and $p_j$ as $b(i, j)$. From the hint, we can think this as a binary tree, so level to level, we have $b(1, 2), b(1, 3) b(2, 3), b(1, 4), b(2, 4), ...,b(i, j)$ (For each point, we either pick to put it the path or not). We know that bitonic tours are tours that start at the leftmost point and go strictly rightward to the rightmost point, and then go strictly leftward back to the starting point. So all points that are left to the $p_j$ are visited by the tour. So when $i < j - 1$, in order to get $b(i,j)$, we need to know $b(i, j-1)$ and $d(j-1, j)$. We have $b(i,j) = b(i, j-1) + d(j-1, j)$. The baseline is $b(1,2) = d(1,2)$, baseline is simply the euclidean distance between $p_1$ and $p_2$. When $i = j - 1$, we know that there must be some path that gets to $p_k$ and from $p_k$ that gets to $p_j$, and $1 < k < j - 1$. So at this time we need to get the shortest path among all of such paths. We can write the dynamic algorithm as 
\begin{flalign*}
&b(1,2) = d(1,2) \\
&b(i,j) = b(i, j-1) + d(j-1, j), \quad i < j - 1 \\
&b(i,j) = min\{b(k, j) + d(k, j)\}, \quad i = j - 1, 1 < k < j - 1
\end{flalign*}
No matter $i < j - 1$ or $i = j - 1$, it takes two for-loops to do the calculation. So the algorithm is $O(n^2)$.
\\ \newline
\textbf{55. The goal in this problem is to give polynomial time approximation scheme (PTAS) for the parallel machine scheduling problem described in problem 35-5 in CLR, for the case that the number of processors/machines $m = 3$, using the following approach:} \\ \newline
\textbf{• First give a dynamic programming algorithm whose time in polynomial in $n$, and $P$, where $P$ is the total aggregate processing times of all of the jobs. So this is a pseudo-polynomial-time algorithm, but not a polynomial time algorithm.} \\ \newline
\textbf{• Then explain how convert this pseudo-poly-time dynamic program into PTAS.}\\ \newline
For each of the task $p_i$, there are only $m = 3$ possible scheduling policies (on machine 1 or 2 or 3), so in all there is $3^n$ feasible solutions. From the hint, we can consider all the feasible solutions as tree, where for each level, a machine decides which job to pick. Each level, there are three nodes (three machines). If 3 nodes (machines) on the same level have the same number of jobs, then prune the one with longer completion time. Let's denote $n$ as the number of jobs and $J_i$ as job $i$. So from $i = 1$ to $i = n$, we can generate level $i$ of the tree from level $i - 1$. Each time, we need to decide which machines can be available first (has shortest completion time) at level $i-1$. Then we need to decide after adding $J_i$, which one has the shortest completion time. In this way, we can generate level $i$ of the tree.  
\\ \newline
In order to give a PTAS, first we need to decide how small the jobs should be before we can ignore them. Short jobs should be short enough so that the error would be small and long jobs should be long enough so we would only have a few long jobs. We pick $T^*$ such that $T^* \leq OPT$. A job $j$ is long if $p_j \geq \epsilon T^*$ and it is short if  $p_j \leq \epsilon T^*$ which is $\leq \epsilon OPT$. Now for the PTAS, 1) we need to find a schedule for the long jobs that runs in at most $(1 + \epsilon)OPT$ time, 2) add short jobs greedily. We need to convert the pseudo-poly-time dynamic program into an algorithm that, for any $T$, to decide which machine executes long jobs, it finds a schedule of the long jobs of length $\leq (1 + \epsilon)T$, if $OPT \leq T$ We find the smallest $T$ that the algorithm can find a schedule of $\leq (1 + \epsilon)T$. If the smallest $T$ is $T^*$, we know that $T^* \leq OPT$. Now we have a schedule for long jobs that has length $\leq (1 + \epsilon)T^* \leq (1 + \epsilon)OPT$. Now we need to decide how to add short jobs. At each level of the tree structure, the algorithm decides that the machine that is first available would be the one that executes a short job. So the time for executing short jobs should be 
\begin{align}
    &\leq \frac{1}{m}\sum_{1\leq k \leq n}p_k + max_1\leq k \leq p_k \\
    & \leq \frac{1}{m}\sum_{1\leq k \leq n}p_k + \epsilon T^*\\
    & = (1 + \epsilon) T^*
\end{align}
$m = 3$ in the inequality above. So we convert the dynamic program into a $(1 + \epsilon)$ approximation algorithm. 



\end{document}
