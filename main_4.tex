\documentclass{article}
\usepackage[utf8]{inputenc}

\title{Assignment 4}
\author{Xiaoting Li (xil139)}
\date{}

\begin{document}

\maketitle

\noindent
\textbf{7. For each of the next 4 algorithms, state whether the algorithm is a polynomial time algorithm, whether the algorithm is a pseudo-polynomial time algorithm, and whether the algorithm is a strongly polynomial time algorithm. Justify your answers.} \\ \newline
Answer: Assume $x_1, x_2, ..., x_n$ are m-bit positive integers. \\
(a) Since we assume $x_1, x_2, ..., x_n$ are m-bit positive integers, it means that we need to multiply each pair of m-bit positive numbers in two for-loops. The time complexity for multiplication is $O(m^{2})$. And since there are two for-loops, so the run-time is $O(n^{2}\cdot m{^2})$. This is polynomial time algorithm since it has nothing to do with the actual numeric value of the input. It only relates to the size of input.\\
(b) Since the second loop is determined by the value of $x_i$ and we assume $x_1, x_2, ..., x_n$ are m-bit positive integers, so the run-time for this question is $O(n\cdot 2^{m}\cdot m{^2})$. This is a pseudo-polynomial time algorithm. Since the run-time is related with the numeric value of the input number. If we use unary coding, then this algorithm can be considered as polynomial time. \\
(c) \\
(d) \\ \newline

\noindent
\textbf{}

\noindent
\textbf{8. Problem 16.4-4 from CLRS.} \\ \newline
Answer: In order to show, $(S,I)$ is a matroid, we need to show that hereditary property and exchange property hold for elements in $I$. \\
1. Hereditary Property \\ 
Ley $X$ and $Y$ be two sets such that $X \in I$ and $Y \subset X$. From the definition of $I$, $X \cap S_i \leq 1$ for $i = 1, 2, ....., k$. If $Y \subset X$ then, $(Y \cap S_i) \in (X \cap S_i)$, so, $(Y \cap S_i) \leq (X \cap S_i) \leq 1$ for all $i$. Then we can say $Y \in I$ since the definition for $I$ holds for $Y$ too. So, the hereditary property holds. \\
2. Exchange Property \\
Let $X$ and $Y$ be two sets such that $X \in I$, $Y \in I$ and $|X| < |Y|$. In order to show that the exchange property holds, we need to show for some $x \in Y \setminus X$, $X \cup \{x\} \in I$.\\
 Assume $|Y \cap S_i| = 1$ and $Y \cap S_i = x$ and $X \cap S_i = \emptyset$ for some $i$. Then, $(X \cup \{x\}) \cap S_i = x$. And for any $j \neq i$, $(X \cup \{x\}) \cap S_j = (X \cap S_j)$. So, now we have $(X \cup \{x\}) \cap S_n \leq 1$ for all $n = 1, 2, ....., k$. Then, we can say  $(X \cup \{x\}) \in I$ and so, the exchange property holds.\\
Since the two properties hold, we can say $(S, I)$ is a matroid.
\noindent
\textbf{}

\end{document}
