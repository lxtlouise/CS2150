\documentclass{article}
\usepackage[utf8]{inputenc}

\title{Assignment 4}
\author{Xiaoting Li (xil139)}
\date{}

\begin{document}

\maketitle

\noindent
\textbf{7. For each of the next 4 algorithms, state whether the algorithm is a polynomial time algorithm, whether the algorithm is a pseudo-polynomial time algorithm, and whether the algorithm is a strongly polynomial time algorithm. Justify your answers.} \\ \newline
Answer: Assume $x_1, x_2, ..., x_n$ are m-bit positive integers. \\
(a) Since we assume $x_1, x_2, ..., x_n$ are m-bit positive integers, it means that we need to multiply each pair of m-bit positive numbers in two for-loops. The time complexity for multiplication is $O(m^{2})$. And since there are two for-loops, so the run-time is $O(n^{2}\cdot m{^2})$. This is polynomial time algorithm since it has nothing to do with the actual numeric value of the input. It only relates to the size of input.\\
(b) Since the second loop is determined by the value of $x_i$ and we assume $x_1, x_2, ..., x_n$ are m-bit positive integers, so the run-time for this question is $O(n\cdot 2^{m}\cdot m{^2})$. This is a pseudo-polynomial time algorithm. Since the run-time is related with the numeric value of the input number. If we use unary coding, then this algorithm can be considered as polynomial time. \\
(c) \\
(d) \\ \newline

\noindent
\textbf{}

\end{document}
