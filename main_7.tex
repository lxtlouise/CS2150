\documentclass{article}
\usepackage[utf8]{inputenc}
\usepackage{amsmath}

\title{Assignment 6}
\author{Xiaoting Li (xil139)}
\date{}

\begin{document}

\maketitle

\noindent
\textbf{14. Prove Lemma 5.4 in http://people.cs.pitt.edu/~kirk/cs2150/matroid-intersect-notes.pdf} \\ \newline
Answer:
We can use induction and strong exchange property to prove this lemma. Let $M' = (E, {S'\in I: |S'| <= |S|})$. So $S$ and $T$ are bases of $M'$. If we take $x\in T\backslash S$, based on strong exchange property, there exists $y\in S\backslash T$ such that $S - y + x \in B$ and $T - x + y \in B$, $B$ is the set of bases of $M'$. It's easy to tell that $S - y + x$ and $T - x + y$ are both independent in $M$. Based on the definition of exchange graph, we have $(y, x)$ as an edge in $D_M(S)$. Then we can keep using induction by replacing $T$ using $T - x + y$, $|S\backslash T| = |S\backslash (T - x + y)| + 1$. Since $|S| = |T|$, in the end we can find all matching pairs in $D_M(S)$, which means we find a perfect matching. So we can conclude that if $S$ and $T$ are independent sets in M with $|S| = |T|$, then there exists a perfect matching between $S\backslash T$ and $T\backslash S$ in $D_M(S)$. \\newline

\noindent \textbf{15.  (a) Problem 29.2-6 from the CLRS text. So you want to given an integer linear
programming formulation that models the bipartite matching problem.} \\ \newline
Answer: For a given graph $G(V, E)$, in order to write a linear program  to solve maximum matching problem we need a variable $x_e$ for each edge $e \in E$. The intended meaning of $x_e$ is whether edge $e$ is in the matching or not. If $e$ is in the matching $x_e = 1$ and if it is not, $x_e = 0$. We would also need a constraint for each vertex $v \in V$ since a vertex in a graph can be in at most one matching. To write these as a linear programming formula we have: \\
\begin{flalign*}
\text{max } &\sum_{e \in E} x_e \qquad \text{s.t.} \\
&\forall v \in V \sum_{e \, \text{adjacent to} \, v} x_e \leq 1 \\
&\forall e \in E, \; x_e \in \{0, 1\}
\end{flalign*}

\noindent \textbf{15.  (b) Consider the relaxed linear program where the integrality requirements are dropped. Explain how to find an integer optimal solution from any rational optimal solution to this relaxed linear program} \\ \newline
Answer: The problem is meaningful only if for each edge of $e$ is assoiated with a weight $w_e$. Same as (a),we need a variable $x_e$ for each edge $e \in E$. The intended meaning of $x_e$ is the percentage of edge $e$ that is used in the matching. Similarly, the linear programming formula is:
\begin{flalign*}
\text{max } &\sum_{e \in E} x_e*w_e \qquad \text{s.t.}\\
&\forall v \in V \sum_{e \, \text{adjacent to} \, v} x_e \leq 1 \\
&\forall e \in E, \; x_e \geq 0
\end{flalign*}
\end{document}
