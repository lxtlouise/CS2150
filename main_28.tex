\documentclass{article}
\usepackage[utf8]{inputenc}

\title{Assignment 28}
\author{Xiaoting Li (xil139) \\
Ziyu Zhang (ziz41) \\
Deniz Unal (des204)}
\date{March 2019}

\begin{document}

\maketitle

\noindent
\textbf{49. Problem 35.2-3 from the CLRS text.} \\ \newline
Answer: From the problem, we learn that at each step, we need to  identify the vertex u that is not on the cycle but whose distance to any vertex on the cycle is minimum. This is similar to finding minimum spanning tree using Prim's Algorithm. In order to prove that this heuristic returns a tour whose total cost is not more than twice the cost of an optimal tour, we set the lower bound as minimum spanning tree. And we need to prove $A(I) \leq 2\cdot LB(I) = 2\cdot MST(I)$. Assume that we have $i$ vertices of current cycle. The cost of the algorithm is $C_{i}$. And assume we have $C_{i} \leq 2 \cdot MST_{i}$, $MST_{i}$ is the cost of minimum spanning tree when there are $i$ vertices. We need to show that for $i+1$, we also have $C_{i+1} \leq 2 \cdot MST_{i+1}$. Let's say for $i+1$, the nearest vertex to $v$ is $u$, $w(v, u)$ is the min weight edge. Using Prim's algorithm, we learn that $MST_{i+1} = MST_{i} + w(v, u)$. Assume the vertex after $v$ is $w$. In order to insert $v$ using the heuristic algorithm, we need to remove $w(v, w)$ and add $w(v, u)$ and $w(u, w)$. So we have $C_{i+1} = C{i} - w(v,w) + w(v, u) + w(u, w)$. Based on triangle inequality, we have $w(v, u) + w(v, w) > w(u, w)$. So we have $C_{i+1} \leq C_{i} + 2\cdot w(v,u) \leq 2\cdot MST_{i} + 2\cdot w(v,u) = 2\cdot MST_{i+1}$. So we can conclude that this heuristic returns a tour whose total cost is not more than twice the cost of an optimal tour.\\ \newline
\textbf{50. Problem 35.2-4 from the CLRS text.}\\ \newline
Answer: As our lower bound, we first compute a bottle neck spanning tree in polynomial time. Then we walk around this bottle neck spanning tree to get a hamiltonian cycle $H$. Since we can travel to a node only once in a hamiltonian cycle, we walk around the bottle neck spanning tree using a short cut, as stated in the hint given in the problem (taking a full walk of the tree and skipping nodes, but without skipping more than two consecutive intermediate nodes). If we skip 2 vertices to add an edge to $H$, by triangle inequality the cost of that edge is smaller than the 3 edges that covering the 2 vertices we skipped. If we call the maximum cost edge in the bottle neck spanning tree $M$, then any edge we add to $H$ should be less than or equal to $3M$. If we call the most costly edge in $H$ $M_H$, then from the previous statement we also know that $M_H \leq 3M$. Let us call the cost of the most costly edge in an \textit{optimal} bottleneck hamiltonian cycle $M_{H_{OPT}}$.  We can always get a spanning tree from the optimal hamiltonian cycle by removing the most costly edge in it, and so the minimum cost of the max cost edge in any spanning tree (including the bottle neck spanning tree) is less than or equal to the cost of the most costly edge in any Hamiltonian cycle. So from that we get $M \leq M_{H_{OPT}}$. And this implies $M_H \leq 3 M_{H_{OPT}}$. By this inequality, we can say that there exists a 3 approximation algorithm for solving the bottle neck traveling salesman problem. \\ \newline
\textbf{51.(a) Problem 35.1-5 from the CLRS text} \\ \newline
Answer: \\ \newline
\textbf{(b) Problem 35.2-2 from the CLRS text} \\ \newline
Answer: \\ \newline
\textbf{(c) Prove that if there is a polynomial time approximation algorithm for the maximum clique problem that has approximation ratio 1000 then there is a polynomial time approximation algorithm with approximation ratio 1.000000001. This is actually a slightly easier problem than problem 35-2 part b in the CLRS text, which I suggest that you look at for inspiration. Note that in some sense this can be viewed as a gap reduction.}\\ \newline
Answer: 

\end{document}
