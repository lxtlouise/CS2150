\documentclass{article}
\usepackage[utf8]{inputenc}

\title{Assignment 6}
\author{Xiaoting Li (xil139) \\
Ziyu Zhang (ziz41) \\
Deniz Unal (des2014)}
\date{January 30, 2019}

\begin{document}

\maketitle

\noindent
\textbf{12. Consider the following problem. The input consists of a directed graph $G = (V, E)$, a designated sink vertex $t \in V$ , a collection $S \subset V$ of source vertices, and a profit $p_v$ for each vertex $v \in S$. A feasible solution is a subset $T$ of $S$ such that there exists a collection of vertex disjoint paths from the elements of $T$ to $t$ in $G$. The objective is to maximize the aggregate profit of the elements of $T$. So think about the following application: the set $S$ is clients, the profits are how much each client is willing to pay for connectivity, and the graph is a computer network that can only support one connection per router, and the goal is to make as much profit as possible by selling collectivity. Give a polynomial time algorithm for this problem. Analyze the run time of the algorithm if it is implemented in a straight-forward way using known algorithms.} \\ \newline
\noindent
Answer: We can reduce this problem to max flow problem. The max flow we need is the maximum matching in a bipartite graph. So we can further reduce this problem to looking for a maximum matching in a bipartite graph using matroid intersection. If we introduce an artificial 'super source' vertex $x$ into our original graph and connect it to each original source vertex $s \in S$ with  directed edges with infinite capacity. We represent each source vertex $s \in S$ with two vertices $s_{in}$ and $s_{out}$. Every incoming edge to $s$ would be coming to $s_{in}$  and every outgoing edge from $s$ would be going from $s_{out}$. And we would connect $s_{in}$ and $s_{out}$ with an edge with capacity equal to weight/profit of $S$. We rewrite the problem using matroid representation, $I_{1} = \{F\subset E | \forall s_{in} \in S_1: |F \cap \delta(s_{in})\leq 1\}$. $I_{2} = \{F\subset E | \forall s_{out} \in S_2: |F \cap \delta(s_{out})\leq 1\}$. $S_1\cup S_2$ are the vertices in $S$ and the intersection $I_{1}\cup I_{2}$ is the set of the matching. And in this problem, what we need to do is find a maximum matching.We can run Edmonds-Karp implementation of Ford Fulkerson method to find the max flow from our artificial source $x$ to the designated sink $t$, since the maximum flow would be equal to maximum matching.


\noindent
\textbf{13.}\\ \newline
\noindent
Answer: The solution coulde be from the source collection S, pick as many number of subset of S such that those subset of S could be connected to sink with the largest time limit that is less or equal than L. And then do this pick up procedure iteratively, so that at the end, we found for each $i \subset [1,L]$. Since the graph have limited number of S, the time for each of the step is poly-time.
\end{document}
