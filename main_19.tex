\documentclass{article}
\usepackage[utf8]{inputenc}
\usepackage{amsmath}
\usepackage{mathtools}

\title{Assignment 19}
\author{Xiaoting Li (xil139) \\
Ziyu Zhang (ziz41) \\
Deniz Unal (des2014)}
\date{March 4, 2019}

\begin{document}

\maketitle

\noindent
\textbf{30.} \\ \newline
\textbf{(a)} $E[y] = Prob(Y\leq L/D) * (L/D) + Prob(Y \geq L/D) * CD$ 
\newline
$Prob(Y\leq L/D) * (L/D) + Prob(Y \geq L/D) * CD \leq Prob(Y\leq L/D) * L + Prob(Y \geq L/D) * CD$ 
\newline
$ Prob(Y\leq L/D) * L + Prob(Y \geq L/D) * CD \leq L + Prob(Y \geq L/D) * CD$
\newline
$ L + Prob(Y \geq L/D) * CD \leq L + Prob(\sum_{i = 1}^{n} Y_i \geq L/D) *CD$
\newline
$ L + Prob(\sum_{i = 1}^{n} Y_i \geq L/D) \leq L + \sum_{i = 1}^{n} Prob[F \geq L/D]$
\newline
$ L + \sum_{i = 1}^{n} Prob[F \geq L/D] \leq L + n* 1/(CDn)^6 *CD$
\newline
$ L + n* 1/(CDn)^6 *CD \leq 20L$
\\ \newline
\textbf{(b)} Since $F = max_iY_i$, so $Prob[F \geq L/D]$ is the probability of $F \geq L/D$. $\sum_e\sum_tProb[F_{e,t} \geq L/D]$ is the probability when the number of packets waiting to cross an edge $e$ at time $t$ is greater than $L/D$ on all of the edges at all the time, which is maximum value of the probability. Since the value of $F_{e,t}$ varies on different edges at different time, so we have $Prob[F \geq L/D] \leq \sum_e\sum_tProb[F_{e,t} \geq L/D]$.\\ \newline
\textbf{(c)} We know $\sum_e\sum_tProb[F_{e,t} \geq L/D]$ means the number of packets waiting to cross edge $e$ at time $t$ and from (b) we learn that $Prob[F \geq L/D] \leq \sum_e\sum_tProb[F_{e,t} \geq L/D]$. So we have the following equations, $(nD)$ is the number of possible choices for the edge $e$ and $(C + D)$ is the number of choices for the time step $t$.
\begin{align*}
&Prob[F\geq L/D] \leq \sum_e\sum_tProb[F_{e,t} \geq L/D] \\
&\leq(CDn)^{12}(nD)(C + D)
\end{align*}\\ \newline
\textbf{(d)} (i) In Bernoulli trials, the result of each trial is either one or zero. In our case, $x_{i,e,t}$ is the random 0/1 variable that is equal to 1 if packet $i$ cross edge $e
$ at time $t$. In addition, $i\in [1, C]$ and each packet is initially delayed by an amount of time that is selected uniformly and independently from the range $[0,C]$ before it starts moving. Since the capacity of each edge is infinite, each packet only needs to wait for the initially assigned delay and travels certain length to get to destination without waiting to any other node's queue. Each $x_{i,e,t}$ is independent. So $x_{i,e,t}$ meets the properties of Bernoulli trial. We can conclude that $x_{1,e,t}, x_{2,e,t}, ... , x_{C,e,t}$ are Bernoulli trials.\\ \newline 
(ii) We learn that for an edge $e$, it is used at most $C$ paths. And since we are given the range of the initial delay time, we know that the probability of $(L/D)$ packets go through this edge at time $t$ is $\big(\frac{1}{C})^{L/D}$.
\begin{align*}
&Prob[F_{e,t} \geq L/D] = \big(_{L/D}^C)\big(\frac{1}{C})^{L/D} \\
&=\frac{C!}{(L/D)!(C - L/D)!}\big(1/C)^{L/D}\\
&\leq \frac{e^{L/D}}{(L/D)^{L/D}}
\end{align*}\newline
\textbf{(e)} When the capacity of each edge is one and using FIFO at each edge to decide which packet goes next if there is congestion means that each $x_{i,e,t}$ is no longer independent. Besides the initially assigned delay and the steps to take to travel to destination, each $x_{i,e,t}$ may be influenced by any other node's queue, which means each $x_{i,e,t}$ is dependent on other packets. So $x_{1,e,t}, x_{2,e,t}, ... , x{C,e,t}$ are not necessarily Bernoulli trials.\\ \newline
\textbf{(f)} We couldn't come up with a scheme, but the goal would be lower $Prob(Y \geq L/D)$, which makes $E[y] = Prob(Y\leq L/D) * (L/D) + Prob(Y \geq L/D) * CD$  lower and $E[y]\leq 2L$.
\\ \newline
\textbf{(h)} \\ \newline

\end{document}
