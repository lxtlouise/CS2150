\documentclass{article}
\usepackage[utf8]{inputenc}

\title{Assignment 19}
\author{Xiaoting Li (xil139) \\
Ziyu Zhang (ziz41) \\
Deniz Unal (des2014)}
\date{March 4, 2019}

\begin{document}

\maketitle

\noindent
\textbf{30.} \\ \newline
\textbf{(a)} \\ \newline
\textbf{(b)} Since $F = max_iY_i$, so $Prob[F \geq L/D]$ is the probability of $F \geq L/D$. $\sum_e\sum_tProb[F_{e,t} \geq L/D]$ is the probability when the number of packets waiting to cross an edge $e$ at time $t$ is greater than $L/D$ on all of the edges at all the time, which is maximum value of the probability. Since the value of $F_{e,t}$ varies on different edges at different time, so we have $Prob[F \geq L/D] \leq \sum_e\sum_tProb[F_{e,t} \geq L/D]$.\\ \newline
\textbf{(c)}\\ \newline
\textbf{(d)} (i) In Bernoulli trials, the result of each trial is either one or zero. In our case, $x_{i,e,t}$ is the random 0/1 variable that is equal to 1 if packet $i$ cross edge $e
$ at time $t$. In addition, $i\in [1, C]$ and each packet is initially delayed by an amount of time that is selected uniformly and independently from the range $[0,C]$ before it starts moving. Since the capacity of each edge is infinite, each packet only needs to wait for the initially assigned delay and travels certain length to get to destination without waiting to any other node's queue. Each $x_{i,e,t}$ is independent. So $x_{i,e,t}$ meets the properties of Bernoulli trial. We can conclude that $x_{1,e,t}, x_{2,e,t}, ... , x{C,e,t}$ are Bernoulli trials.\\ \newline
\textbf{(e)} When the capacity of each edge is one and using FIFO at each edge to decide which packet goes next if there is congestion means that each $x_{i,e,t}$ is no longer independent. Besides the initially assigned delay and the steps to take to travel to destination, each $x_{i,e,t}$ may be influenced by any other node's queue, which means each $x_{i,e,t}$ is dependent on other packets. So $x_{1,e,t}, x_{2,e,t}, ... , x{C,e,t}$ are not necessarily Bernoulli trials.\\ \newline
\textbf{(f)} \\ \newline
\textbf{(h)} \\ \newline

\end{document}
