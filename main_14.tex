\documentclass{article}
\usepackage[utf8]{inputenc}

\title{Assignment 14}
\title{Assignment 13}
\author{Xiaoting Li (xil139) \\
Ziyu Zhang (ziz41) \\
Deniz Unal (des2014)}
\date{}


\begin{document}

\maketitle

\noindent
\textbf{22. Consider the problem of finding the largest $k$ numbers in sorted order from a list of n numbers (see problem 9-1) in the text. Consider the following algorithm: you consider the numbers one by one, maintaining an auxiliary data structure of the largest $k$ numbers seen to date. We get various algorithms depending on what the auxiliary data structure is and how one searches and updates it. For each of the following variations give the worst-case time complexity as a function of $n$ and $k$. For each of the following variations give the average-case time complexity as a function of $n$ and $k$ under the assumption that each input permutation is equally likely.}\\ \newline
\textbf{(a) The auxiliary data structure is an ordered list and you use linear search starting from the end that contains the largest number} \\ \newline
Answer: Let $S_i$ be the number at position $i$ in the list. Let $X_i$ be an indicator random variable. If $S_i$ is compared with elements in the auxiliary data structure, $X_i$ is 1, else, $X_i$ is 0. Since we use linear search starting from the end that contains the largest number, we have to compare the $S_i$ with elements in the auxiliary data structure every time. There are two parts of the total comparisons. The first part is the sorting the first time we have $k$ elements in the auxiliary data structure. The second part is to compare $S_i$ with numbers in the auxiliary data structure to find the correct position for $S_i$ in the auxiliary data structure. So the total number of comparisons is $E(X) = (1+k)k/2 + \sum_{1}^{n}E(X_i) = (1+k)k/2 + klg(n)$.\\ \newline
\textbf{(b) The auxiliary data structure is an ordered list and you use linear search starting from the end that contains the smallest number} \\ \newline
Answer: Let $S_i$ be the number at position $i$ in the list. Let $X_i$ be an indicator random variable. If $S_i$ is compared with elements in the auxiliary data structure, $X_i$ is 1, else, $X_i$ is 0. There are three parts when we compute the time complexity. The first part is the sorting the first time we have $k$ elements in the auxiliary data structure. The second part is the comparison with the smallest number in the auxiliary data structure. Since we use linear search starting from the end that contains the smallest number, if $S_i$ is smaller than the smallest number in the auxiliary data structure, it means there is no need to compare $S_i$ with numbers in the auxiliary data structure. The third part is when $S_i$ is larger than the smallest number, then we need to compare $S_i$ with numbers in the auxiliary data structure to find the correct position for $S_i$ in the auxiliary data structure. So the total number of comparisons is $E(X) = (1+k)k/2 + (n - k) + \sum_{k+1}^{n}E(X_i)$. Since the probability of $S_i$ to be one of the $k$ largest numbers is $k/i$, so $E(X) = (1+k)k/2 + (n - k) - klg(k) + klg(n)$.\\ \newline
\textbf{(c) The auxiliary data structure is a balanced binary search tree and you use standard log time search, insert and delete operations} \\ \newline
Answer: Let $S_i$ be the number at position $i$ in the list. Let $X_i$ be an indicator random variable. If $S_i$ is compared with elements in the auxiliary data structure, $X_i$ is 1, else, $X_i$ is 0. Since we use linear search starting from the end that contains the largest number, we have to compare the $S_i$ with elements in the auxiliary data structure every time. There are two parts of the total comparisons. The first part is the sorting the first time we have $k$ elements in the auxiliary data structure. The second part is to compare $S_i$ with numbers in the auxiliary data structure to find the correct position for $S_i$ in the auxiliary data structure. Since this is a balanced binary search tree, the first part is $lgk$. So the total number of comparisons is $E(X) = lgk + \sum_{1}^{n}E(X_i) = lgk + nlgk = (n+1)lgk$. \\ \newline
\textbf{(d) The auxiliary data structure is a balanced binary search tree and you use standard log time insert and delete operations, but you start your search from the smallest item in the tree} \\ \newline
Answer: Let $S_i$ be the number at position $i$ in the list. Let $X_i$ be an indicator random variable. If $S_i$ is compared with elements in the auxiliary data structure, $X_i$ is 1, else, $X_i$ is 0. There are three parts when we compute the time complexity. The first part is the sorting the first time we have $k$ elements in the auxiliary data structure. The second part is the comparison with the smallest number in the auxiliary data structure. Since we use linear search starting from the end that contains the smallest number, if $S_i$ is smaller than the smallest number in the auxiliary data structure, it means there is no need to compare $S_i$ with numbers in the auxiliary data structure. The third part is when $S_i$ is larger than the smallest number, then we need to compare $S_i$ with numbers in the auxiliary data structure to find the correct position for $S_i$ in the auxiliary data structure. So the total number of comparisons is $E(X) = lgk + (n - k) + \sum_{k+1}^{n}E(X_i) = lgk + (n - k) + (n - k)lgk = (n + 1 - k)lgk + (n - k)$.\\ \newline
\textbf{23. Assume a router sees a stream of IP packets from two different sources. So the router sees a packet, and then can do some minimal computation, then forwards the packet, sees the next packet, etc. The router is trying to determine the similarity of the destination IP addresses for the two different sources while using very little space. Let $A$ be the collection of destination IP addresses for the first source, and B be the collection of destination IP addresses for the second source. Assume that we have a hash function $h: I -> R$ that maps IP addresses to integers, where the range is sufficiently large that the probability of a collision is negligible. Assume that that the hash function h uniformly distributes the collection $I$ of possible IP addresses over the range $R$, and that each source IP is picked uniformly and independently from the domain of possible IP addresses.} \\ \newline
\textbf{(a) First consider a naive approach. Let a be a random element of $A$ and $b$ a random element $B$. If $A = B$, what is the probability $h(a) = h(b)$? If A and B are disjoint, what is the probability that $h(a) = h(b)$? Calculate the probability that $h(a) = h(b)$ in terms of $|A|$, $|B|$, $|A\cup B|$, and $|A\cap B|$.} \\ \newline
\textbf{(b) Now we turn to something a bit more sophisticated. Let $h_m(A)$ be the minimum integer k such there is an element $x$ of $A$ where $h(x) = k$. Let $h_m(B)$ be the minimum integer k such there is an element x of B where $h(x) = k$. If $A = B$, what is the probability that $h_m(A) = h_m(B)$? If $A$ and $B$ are disjoint, what is the probability that $h_m(A) = h_m(B)$? Remember that we are assuming that the probability of a collision is negligible.}\\ \newline
\textbf{(c) Calculate the probability that $h_m(A) = h_m(B)$ in terms of $|A|$, $|B|$, $|A\cup B|$, and $|A\cap B|$.}\\ \newline
\textbf{(d) Explain how the probability that $h_m(A) = h_m(B)$ is an estimate of the similarity of $A$ and $B$.} \\ \newline
\textbf{(e) Explain how to maintain $h_m(A) = h_m(B)$ using constant space, and constant time per IP packet.} \\ \newline
\textbf{24. Assume you have a source of random bits. So in one time unit, this source will produce one random bit (that is 1 with probability 1/2 independent of other bits). Consider the problem of outputting a random permutation of the integers from 1 to $n$. So each of the n! permutations should be produced with probability exactly $1/n!$.} \\ \newline
\textbf{(a) Give an algorithm to solve this problem and show that the expected time of the algorithm is $O(nlogn)$.This includes both the time that your algorithm takes, plus 1 unit of time for each random bit used.} \\ \newline
\textbf{(b) Now assume that there is a limited source of at most $n^2$ random bits. Show that there is no algorithm that can solve the problem using expected time $O(n^2)$.}


\end{document}
