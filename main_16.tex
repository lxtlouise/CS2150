\documentclass{article}
\usepackage[utf8]{inputenc}

\title{Assignment 16}
\author{Xiaoting Li (xil139) \\
Ziyu Zhang (ziz41) \\
Deniz Unal (des2014)}
\date{}

\begin{document}

\maketitle

\noindent
\textbf{26. Consider the Karger, Klein and Tarjan (KKT) randomized linear-time minimum spanning tree algorithm that we considered in class. Let G be some arbitrary graph. Let $\tau$ be the collection of spanning trees of $G$. So each element of $\tau$ is a spanning tree of $G$. Assume that each edge in $G$ independently and uniformly at random is assigned a random real in the range [0, 1]. Let T be the resulting (random variable) tree produced by running the KKT algorithm on this weighted graph. Prove or disprove that for each $S\in \tau$, it is the case that $Prob[S=T]=1/|T|$. } \\ \newline
Answer: In this problem, we need to prove that for each $S\in \tau$, it is the case that $Prob[S=T]=1/|T|$. It actually means that we need to prove there's a unique minimum spanning tree when we are given a graph in which each edge is assigned a random real in the range [0, 1] independently and uniformly. We can prove this using contradiction. Assume we have two distinct minimum spanning tree for this weighted graph, $T_1 = (V, E_1)$ and $T_2 = (V, E_2)$ (Suppose both $T_1$ and $T_2$ are minimum spanning trees produced by running KKT algorithm on this weighted graph). Since they are distinct minimum spanning trees, so we have $|E_1 - E_2| > 0$. There exist an edge $e$ is in $E1 - E2$ that causes cycle in $T_2$. Since $T_2$ is generated by KKT algorithm, it means $e$ is a heavy edge since KKT discards heavy edges when it produces minimum spanning tree. However, we learn that $e$ is in $E1 - E2$. $E_1$ are edges belong to $T_1$ and $T_1$ is also a minimum spanning tree produced by KKT. $e$ is heavy edge and it is in $T_1$, but $T_1$ is a minimum spanning tree produced by KKT(KKT discards heavy edges when it produces minimum spanning tree). We have a contradiction. Therefore, we can conclude that KKT only produces a unique minimum spanning tree. And since $\tau$ is the collection of spanning trees of $G$, so we have $Prob[S=T]=1/|T|$.\\ \newline
\textbf{27. Recall Karger’s algorithm for min-cut. One pass used $n - 2$ edge contractions and computed an optimal cut with probability $\Omega(1/n^2$. Here $n$ is the number of vertices in the initial graph $G$. Thus $n^2$ passes use $n^3$ edge contractions, and compute the optimal cut with probability $\Omega(1)$. The goal here is to compute the optimal cut with probability $\Omega(1)$ using $o(n^3)$ edge contractions. It is sufficient to design an algorithm that uses $O(n)$ edge contractions and computes the optimal cut with probability $\omega(1/n^2)$. To accomplish this consider the following way to modify a pass of Karger’s: 
\begin{itemize}
  \item Contract $j$ random edges to get a new multi-graph $H$.
  \item repeat k times: \\
  - Contract $n-j-2$ random edges in $H$\\
  - If the resulting cut is cheaper than any previously seen cut then make this the smallest cut so far
 \item Output the smallest cut found
\end{itemize}
Note that the repeat loop always starts with $H$, the multigraph that results from the first $j$ random edge contractions. So conceptually the output is the best of $k$ cuts, but each of these $k$ cuts are not independent in the sense that have the first $j$ edge contractions in common. Your goal is to find values of $j$ and $k$ such that that this algorithm uses $O(n)$ edge contractions and computes the optimal cut with probability $\omega(1/n^2)$ Show your work.} \\ \newline
Answer:


\end{document}
