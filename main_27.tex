\documentclass{article}
\usepackage[utf8]{inputenc}

\title{Assignment 27}
\author{Xiaoting Li (xil139) \\
Ziyu Zhang (ziz41) \\
Deniz Unal (des204)}
\date{March 29 2019}

\begin{document}
\maketitle
\noindent
\textbf{45 - 46. In the Vertex Cover Decision problem (VCD) the input is a graph $G$ and an integer $k$, and the output is 1 if G has a vertex cover of size $k$, and 0 otherwise. In the Vertex Cover Optimization problem (VCO) the input is a graph $G$, and the output is a minimum cardinality vertex cover of $G$.} \\ \newline
\textbf{(a) Show that if VCD has a polynomial time algorithm, then VCO has a polynomial time algorithm.} \\ \newline
Answer: We assume that we have a poly time solution for VCD. We can start with $k =$ number of vertices in $G$. Then we would run our algorithm for VCD recursively to see if there's a vertex cover of size $k$. We would decrement $k$ and continue for each value of $k$ until it is 0. The smallest value of $k$ that would output 1 (there is a vertex cover of size $k$) from VCD would be the size of min cardinality vertex cover in $G$. We implement another algorithm based on VCD, whenever it would output a 1, instead, it would return the set of vertices picked (This is stil poly time because VCD is poly time). Then we would just run this algorithm with the size of min cardinality vertex cover found from VCD and so it would return the set of vertices picked. This is the output needed for VCO. So, if we can have a poly time solution for VCD we also have a poly time solution for VCO.\\ \newline
\textbf{(b) Show that if VCD has a fixed parameter tractable algorithm in the parameter $k$, then VCO has a fixed parameter tractable algorithm in the parameter $\ell$, which is the cardinality of the minimum vertex cover.} \\ \newline
Answer: \\ \newline
\textbf{47. Problem 35-1 parts a through e from CLRS text} \\ \newline
Answer: \\ \newline
\textbf{48. Problem 35-5 parts a, b and d from the CLRS text} \\ \newline
Answer: \\ \newline
\end{document}
